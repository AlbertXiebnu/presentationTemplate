\documentclass[table]{beamer}
\mode<presentation>
{
  \usetheme{Berkeley}
}

% 设定英文字体
\usepackage{fontspec}
\setmainfont{Arial}
%\setmainfont{Times New Roman}
\setsansfont{Arial}
\setmonofont{Courier New}

% 设定中文字体
\usepackage[BoldFont,SlantFont,CJKchecksingle,CJKnumber]{xeCJK}
\setCJKmainfont[BoldFont={Adobe Heiti Std},ItalicFont={Adobe Kaiti Std}]{Adobe Song Std}

\usepackage{setspace}
\usepackage{booktabs}
\usepackage{colortbl,xcolor}
\usepackage{hyperref}

% 插入图片
\usepackage{graphicx}
% 指定存储图片的路径(当前目录下的figures文件夹)
\graphicspath{{figures/}}

% 可能用到的包
\usepackage{amsmath,amssymb}
\usepackage{multimedia}
\usepackage{multicol}

% item逐步显示时,使已经出现的item、正在显示的item、将要出现的item呈现不同颜色
\def\hilite<#1>{
 \temporal<#1>{\color{gray}}{\color{blue}}
    {\color{blue!25}}
}

% 在表格、图片等得标题中显示编号
\setbeamertemplate{caption}[numbered]

